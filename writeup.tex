\documentclass[twoside]{article}

\usepackage[sc]{mathpazo} % Use the Palatino font
\usepackage[T1]{fontenc} % Use 8-bit encoding that has 256 glyphs
\linespread{1.05} % Line spacing - Palatino needs more space between lines
\usepackage{microtype} % Slightly tweak font spacing for aesthetics

\usepackage[hmarginratio=1:1,top=32mm,columnsep=20pt]{geometry} % Document margins
\usepackage{multicol} % Used for the two-column layout of the document
\usepackage[hang, small,labelfont=bf,up,textfont=it,up]{caption} % Custom captions under/above floats in tables or figures
\usepackage{booktabs} % Horizontal rules in tables
\usepackage{float} % Required for tables and figures in the multi-column environment - they need to be placed in specific locations with the [H] (e.g. \begin{table}[H])
\usepackage{hyperref} % For hyperlinks in the PDF

%----------------------------------------------------------------------------------------
%	TITLE SECTION
%----------------------------------------------------------------------------------------

\title{\vspace{-15mm}\fontsize{24pt}{10pt}\selectfont\textbf{World Cup 2014 Predictor}} % Article title

\author{
\large
\textsc{Britt Cyr}
\\ % Your institution
\normalsize \href{mailto:cyrbritt@gmail.com}{cyrbritt@gmail.com} % Your email address
\vspace{-5mm}
}
\date{}

%----------------------------------------------------------------------------------------

\begin{document}

\maketitle % Insert title

%----------------------------------------------------------------------------------------
%	ARTICLE CONTENTS
%----------------------------------------------------------------------------------------

\begin{multicols}{2} % Two-column layout throughout the main article text


%------------------------------------------------

\section{Overview}
The layout of this project is first creating rating for every qualified team
using the past data of match results for the past few years. Once I have the ratings for
all the teams, I precompute the probable result if every pair of teams would meet in
the tournament. Once I have done this, I run many simulations of the tournament.

\section{Team Ratings}
To get ratings for all the teams, I first load all the data provided in the csv. Each
score is then reduced to a score differential. 
I do not use all games in the file though. I ignore those with extremely large goal
differential since the exact score differential is then likely inaccurate.
I run a least squares algorithm to 
get a rating for every team. I run this with a variable home field advantage parameter
that optimizes the home advantage of a normal game while attempting to minimize the
error of the least squares algorithm. The result of this step is to get a numeric rating
for every team. For example, Brazil may be 9.5 and Germany may be 10.7 which would indicate
that if the teams played on a neutral site, the model would predict that Germany wins
by 1.2 goals on average.

\section{Probable Results}
I next compute the probable outcome of a match between every possible pair of teams.
I do this only once since it will be the same across all simulations. For a pair of teams,
I take each of their ratings into consideration. I find all games that would be similar. That
means that the ratings of the old game are close to the ratings of the current game. So
if the first team's rating is 10.0, then I would look for games with the first team had a
rating of approximately 10.0 and also require the rating of the second team to be similar.
I aggregate all games that are similar to the current matchup. The 
win/lose/draw probabilities are the
probabilities of the outcomes uniformly across all these similar matches.

\section{Simulation}
In the simulation section I use the computed match probabilities from before. I run the
tournament choosing the results of each match randomly from the projected results precomputed.
The results calculator only gives an outcome, not a goal differential, so points ties are
broken randomly in the group stage. The simulation then requires a winner in the knockout stage.
The output probabilities for every event are the distribution of results for the simulation.

\section{Code}
The code is hosted on \href{https://github.com/brittcyr/finna-be-octo-computing-machine}{github}
I used python and required numpy for the team ratings section and random for simulation.


\end{multicols}

\end{document}